\documentclass[a4paper]{article}
\usepackage[utf8]{inputenc}
\usepackage{geometry}
\usepackage[table,xcdraw]{xcolor}
\usepackage{tabularx}
\geometry{landscape, margin=1cm}

\title{Cronograma del Hackathon de Fin de Semana}
\author{Federico Cabanilla}
\date{\today}

\begin{document}

\maketitle

\section{Sábado}

\begin{table}[]
\begin{tabularx}{\textwidth}{|l|X|}
\hline
\rowcolor[HTML]{BBDEFB} 
\textbf{Hora} & \textbf{Actividad} \\ \hline
16:00 - 16:25 & Revisar los requerimientos: Leer detalladamente cada punto de los requerimientos e identificar los posibles desafíos. \\ \hline
16:32 - 16:57 & Diseño del modelo de la base de datos: Definir qué entidades necesitarás y cómo se relacionan entre sí. \\ \hline
17:04 - 17:29 & Instalación de MongoDB y configuración inicial: Descargar e instalar MongoDB, y configurar el entorno para su uso. \\ \hline
17:36 - 18:01 & Crear la base de datos y las colecciones necesarias: Usar MongoDB para crear la base de datos y las colecciones que diseñaste anteriormente. \\ \hline
\end{tabularx}
\end{table}

\section{Domingo}

\begin{table}[]
\begin{tabularx}{\textwidth}{|l|X|}
\hline
\rowcolor[HTML]{BBDEFB} 
\textbf{Hora} & \textbf{Actividad} \\ \hline
10:00 - 10:25 & Revisar el trabajo del día anterior: Comprobar el trabajo realizado el día anterior y verificar que no haya errores. \\ \hline
10:32 - 10:57 & Crear una API de prueba: Utilizar Express.js para crear una API de prueba con rutas básicas. \\ \hline
11:04 - 11:29 & Prueba de las rutas de la API: Usar Postman o una herramienta similar para probar las rutas de la API y verificar que funcionan correctamente. \\ \hline
\end{tabularx}
\end{table}

\end{document}
