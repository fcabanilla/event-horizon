\documentclass[a4paper]{article}
\usepackage[utf8]{inputenc}
\usepackage[margin=2.5cm]{geometry}

\title{Etapa de Análisis: Sistema de Gestión de Eventos}
\author{Federico Cabanilla ft. ChatGPT}
\date{\today}

\begin{document}

\maketitle

\section{Revisión de Requerimientos}
Revisa los requerimientos capturados durante la etapa de requerimientos. Asegúrate de comprender completamente cada requerimiento y su contexto.

\section{Descomposición de Requerimientos}
Analiza los requerimientos y descompónlos en componentes más pequeños y manejables. Esto te ayudará a entender mejor las funcionalidades específicas que deben ser implementadas.

\section{Identificación de Componentes}
Identifica los principales componentes o módulos del sistema y cómo se relacionan entre sí. Considera la arquitectura y los patrones de diseño que serán más adecuados para tu aplicación.

\section{Diseño de la Base de Datos}
Diseña la estructura de la base de datos. Define las entidades, atributos y relaciones, y crea un modelo de base de datos que cumpla con los requerimientos.

\section{Diseño de la Interfaz de Usuario}
Diseña la interfaz de usuario (UI) teniendo en cuenta los flujos de trabajo de los usuarios y los requerimientos funcionales. Crea wireframes o prototipos para visualizar cómo será la interfaz.

\section{Definición de Reglas de Negocio}
Identifica y documenta las reglas de negocio que deben ser aplicadas en tu aplicación. Estas reglas determinarán cómo se comportará el sistema y cómo se realizarán ciertas acciones o validaciones.

\section{Identificación de Servicios Externos}
Identifica los servicios externos con los que tu aplicación necesitará interactuar, como APIs de terceros. Define cómo se integrarán estos servicios en tu sistema.

\section{Definición de Casos de Uso}
Define los casos de uso basados en los requerimientos y las funcionalidades del sistema. Un caso de uso describe una interacción entre un usuario y el sistema para lograr un objetivo específico.

\section{Validación y Aprobación}
Revisa y valida tus documentos y diseños con los stakeholders y el equipo de desarrollo. Asegúrate de que todos estén de acuerdo con la solución propuesta.

\end{document}
